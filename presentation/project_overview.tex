\documentclass[beamer]{beamer}
\usetheme{Ilmenau}
\usepackage[utf8]{inputenc}
\usepackage{graphicx}
 \usepackage{caption}
 
\usepackage{natbib}

\usepackage{appendixnumberbeamer}


\usecolortheme{beaver}
% Define UT Dallas colors 
\definecolor{utdOrange}{HTML}{e87500} 
\definecolor{utdGreen}{HTML}{154734}
\definecolor{utdTeal}{HTML}{5fe0b7}
\definecolor{utdGrey}{HTML}{A8A49E}

% Apply the colors to the beamer color palettes
% This is an example; you may need to experiment with different palette names
\setbeamercolor*{palette primary}{bg=utdGreen, fg=black}
\setbeamercolor*{palette secondary}{bg=utdOrange, fg=black}
\setbeamercolor*{palette tertiary}{bg=utdGrey, fg=black}
\setbeamercolor*{palette quaternary}{bg=utdTeal, fg=black}

\setbeamercolor{institute in head/foot}{fg=black}
\setbeamercolor{title}{fg=white}
\setbeamercolor{frametitle}{fg=utdGreen}
\setbeamercolor{title in head/foot}{fg=black}
\setbeamercolor{itemize item}{fg=utdGreen}
\setbeamertemplate{itemize item}[square]

\setbeamercolor{itemize subitem}{fg=utdOrange}
\setbeamertemplate{itemize subitem}[triangle]

\setbeamertemplate{caption}[numbered]

\captionsetup[figure]{labelfont={color=utdGreen}}

\makeatletter
%change the footline template to include frame numbers
\defbeamertemplate*{footline}{myminiframes theme}
  {%
    \begin{beamercolorbox}[colsep=1.5pt]{upper separation line foot}
    \end{beamercolorbox}
    \begin{beamercolorbox}[ht=2.5ex,dp=1.125ex,%
      leftskip=.3cm,rightskip=.3cm plus1fil]{author in head/foot}%
      \leavevmode{\usebeamerfont{author in head/foot}\insertshortauthor}%
      \hfill%
      {\usebeamerfont{institute in head/foot}\usebeamercolor[fg]{institute in head/foot}\insertshortinstitute}%
    \end{beamercolorbox}%
    \begin{beamercolorbox}[ht=2.5ex,dp=1.125ex,%
      leftskip=.3cm,rightskip=.3cm plus1fil]{title in head/foot}%
      {\usebeamerfont{title in head/foot}\insertshorttitle\hfill \insertframenumber/\inserttotalframenumber}%<-here
    \end{beamercolorbox}%
    \begin{beamercolorbox}[colsep=1.5pt]{lower separation line foot}
    \end{beamercolorbox}
  }
\makeatother

%change look of sections in ToC
\defbeamertemplate*{section in toc}{mysections in toc}
{\leavevmode ---\,\inserttocsection\par}

%change look of subsections in ToC
\defbeamertemplate*{subsection in toc}{mysections in toc}
{\leavevmode\leftskip=2.5em --\,\inserttocsubsection\par}

\title{Riverine Discharge Changepoint Detection}
\subtitle{An exploration of Sequential Bayesian Onlince Change Point Detection (BOCPD)}
\institute{School of Economics Political and Policy Sciences, UT Dallas}
\author{Mae Hutchison}
\date{December 2, 2025}


\setcounter{framenumber}{0}

\begin{document}

{
\begin{frame}[plain, noframenumbering]
\maketitle
\end{frame}
}

\begin{frame}
\frametitle{Table of Contents}
\tableofcontents 
\end{frame}

\section{Introduction}

\begin{frame}
\frametitle{USGS riverine gage data}
The U.S. Geological Survey (USGS) river gage data provides continuous monitoring of streamflow conditions. 

\begin{figure}[h!] 
    \centering 
    \includegraphics[width=0.5\linewidth]{gage_map.png}
    \caption{Three hydrologically connected gages in Kerr County, TX}
    \label{fig:gage_map}
\end{figure}

\end{frame}

\begin{frame}<beamer:0>[noframenumbering]
\frametitle{Local flow conditions}
\begin{itemize}
	\item These sites are directly connected by the stream network
	\item Water flows from the northernmost gage site (USGS-08166140) through the central gage site (USGS-08166200) to the southernmost gage site (USGS-08166250)
	\item Due to Kerr County’s proximity to the gulf, weather systems frequently move from south to north
\end{itemize}

\end{frame}

\begin{frame}<beamer:0>[noframenumbering]
\frametitle{Local flow conditions}
\begin{figure}[h!] 
    \centering 
    \includegraphics[width=0.7\linewidth]{discharge_summer_2025} 
    \caption{Flow Discharge in Summer 2025. Data retrieved from USGS API \citep{dataRetrieval}} 
    \label{fig:discharge2025} 
\end{figure}

\end{frame}

\section{Flow Model}
\begin{frame}
\frametitle{Distribution}
The Log-Pearson Type III distribution is widely used in hydrology to model 
peak streamflows \citep{England2018Bulletin17C}. It applies a Pearson Type III distribution to the 
log-transformed discharge data.

Let \(Q\) be the daily peak discharge. Define:

\[X = \log_{10}(Q)\]

\[X \sim \text{PearsonIII}(\kappa, \theta, \tau)\]

\end{frame}

\begin{frame}<beamer:0>[noframenumbering]
\frametitle{Probability Density Function (PDF)}
The PDF of the Pearson Type III distribution for \(X\) is:

\[
f(x|\kappa, \theta,\tau) = \frac{(x - \tau)^{\kappa - 1} e^{-(x - \tau)/\theta}}
            {\theta^{\kappa} \Gamma(\kappa)}, \quad x > \tau
\]

Where:  
\begin{itemize}

\item \( \kappa > 0 \) is the shape parameter  
\item \( \theta > 0 \) is the scale parameter  
\item \( \tau \) is the location parameter  
\item \( \Gamma(\cdot) \) is the gamma function
\end{itemize}
\end{frame}

\begin{frame}<beamer:0>[noframenumbering]
\frametitle{Estimating Parameters}
The parameters can be estimated using the first three L-moments of past observed data (mean $\mu$, standard deviation $\sigma$, skewness $\gamma$).
\begin{itemize}
\item shape $\kappa = (2/\gamma)^2$
\item scale $\theta = 2\sigma/\gamma $
\item location $\tau = \mu - \theta \cdot \beta$
\end{itemize}
\end{frame}

\section{Analytical Solution}
\begin{frame}
\frametitle{Bayesian Framework}
For this solution, we use a simplified model where $X \sim \Gamma(\alpha,\beta)$  with shape parameter $\alpha$
and rate parameter $\beta$ (inverse of scale).

With a Gamma-Gamma conjugage prior, the likelihood is Gamma and the conjugate prior for the rate 
parameter $\beta$ (given $\alpha$) is also Gamma.
\begin{itemize}
 \item Prior: $\beta \sim \Gamma(a_0, b_0)$   
 \item Likelihood: $x_i | \beta \sim \Gamma(\alpha, \beta)$
 \item Analytical posterior (with $n$ observed data points): $\beta | x_{1:n}\sim \Gamma(a_0+n\alpha, b_0+\sum_{i=1}^n x_i)$
\end{itemize}
\end{frame}

\begin{frame}<beamer:0>[noframenumbering]
\frametitle{Online Update}
With online updates, we can only see past data at each timestep. When a new observation $x_t$ arrives:
\begin{itemize}
\item update sufficent statistics
	\begin{itemize}
		\item $n \gets n+1$
		\item $S \gets S+x_t$
	\end{itemize}
\item Posterior becomes $\beta|x_{1:n} \sim \Gamma(a_0+n\alpha, b_0 +S)$
\end{itemize}
\end{frame}


\begin{frame}<beamer:0>[noframenumbering]
\frametitle{Changepoint Detection}
Using \citet{AdamsMacKay2007} BOCPD framework,
\begin{itemize}
	\item Maintain a run-length distribution $\Pr(r_t|x_{1:t})$ where $r_t = \text{number of observations since last changepoint}$
	\item For each possible run length, compute predictive probability $\Pr (x_t | r_t) = \int \Pr(x_t|\beta) \Pr(\beta | \text{data in run}) \delta \beta$
\end{itemize}
\end{frame}

\begin{frame}<beamer:0>[noframenumbering]
\begin{figure}[h!] 
    \centering 
    \includegraphics[width=0.5\linewidth]{runlength_probs.png} 
    \caption{Runlength Probabiliites, adapted from \citet{AdamsMacKay2007}}
    \label{fig:runlegthprobs} 
\end{figure}
\end{frame}

\begin{frame}
\frametitle{Analytical Results}
\begin{figure}
\centering
\includegraphics[height=0.7\textheight]{bocpd_cojugate_cpp_withData.png}
\caption{Results of analytical solution on Kerr County streamgage data}
\end{figure}
\end{frame}


\section{Numerical Solution}

\begin{frame}<beamer:0>[noframenumbering]
\frametitle{Stan Model}
\begin{figure}
\centering
\includegraphics[height=0.7\textheight]{stan_model.png}
\caption{Stan model with custom target function}
\end{figure}
\end{frame}

\begin{frame}
\frametitle{Sampler}
For each time step:
\begin{itemize}
\item sample for each run length before current timestep
\item calculate the predictive probability of the observed posterior
\item calculate the run length probability
\item calculate changepoint probability {\small  \[
P(r_t = 0 \mid x_{1:t}) \propto \sum_{r_{t-1}} P(r_t = 0 \mid r_{t-1}) \cdot P(x_t \mid r_{t-1}, x_{1:t-1}) \cdot P(r_{t-1}, x_{1:t-1})
\]}
\end{itemize}
\end{frame}

\begin{frame}
\frametitle{Numerical Results (simplified version)}
\begin{figure}
\centering
\includegraphics[height=0.7\textheight]{numerical_changepoints.png}
\caption{Change in parameter points on Kerr County streamgage data}
\end{figure}
\end{frame}

\section{Ensemble Algorithm}
\begin{frame}
\frametitle{Rbeast}
Rbeast is an R package that uses Bayesian Estimator of Abrupt change, Seasonal change, and Trend (BEAST) \citep{ZHAO2019}

Combines:
\begin{itemize}
\item Bayesian changepoint detection for time series.
\item Ensemble modeling to improve robustness and uncertainty quantification.
\end{itemize}
\end{frame}

\begin{frame}<beamer:0>[noframenumbering]
\frametitle{Ensemble Components}
Each model in the ensemble estimates:
\begin{itemize}
\item Number and location of changepoints.
\item Segment-specific parameters (mean, trend, seasonality).
\end{itemize}

Models differ by:
\begin{itemize}
	\item Prior settings (e.g., hazard rate, variance).
	\item Sampling seeds.
	\item Possibly different likelihood structures.
\end{itemize}
\end{frame}

\begin{frame}<beamer:0>[noframenumbering]
\frametitle{Bayesian Framework}
For each model:
\begin{itemize}
\item sample posterior over changepoints and parameters:
{\tiny $\Pr(changepoints,params|data)\Pr(changepoints, params data)\Pr(changepoints,params|data)$}

\item Can be computed analytically or computed via MCMC depending on likelihood structure.
\end{itemize}

\end{frame}


\begin{frame}<beamer:0>[noframenumbering]
\frametitle{Ensemble Aggregation}
Combines posterior samples from all models.  

Compute:  
\begin{itemize}
	\item Posterior inclusion probability for each time point being a changepoint.
	\item Credible intervals for trends and seasonal components.
\end{itemize}

This reduces sensitivity to prior choices and improves stability.
\end{frame}

\begin{frame}
\frametitle{EA Results}
\begin{figure}
\centering
\includegraphics[height=0.7\textheight]{beast_changepoints.png}
\caption{Rbeast calculated changepoints on Kerr County streamgage data}
\end{figure}
\end{frame}

\section{Future Work}
\begin{frame}
\frametitle{Future Work}
Improve numerical solution with particle filtering
\begin{itemize}
\item Save samples from previous run lengths 
\item Each particle represents the current rung length, segment parameters, and weight
\item Compute likelihood of new observation under particle's parameters
\item Update and normalize weights 
\item Resample particles after certain metrics are triggered (e.g., effective sample size drops).
\end{itemize}
\end{frame}

\appendix
\begin{frame}[allowframebreaks]
        \frametitle{References}
        \bibliographystyle{apalike}
        \bibliography{references.bib}
\end{frame}



\end{document}